\chapter{Introduction}
\label{sec:introduction}

%%%%%%%%%%%%%%%%%%%%%%%%%%%%%%%%%%%%%%%%%%%%%%%%%%%%%%%%%%%%%%%%%%%%%%%%%%%%%%
%%%%%%%%%%%%%%%%%%%%%%%%%%%%%%%%%%%%%%%%%%%%%%%%%%%%%%%%%%%%%%%%%%%%%%%%%%%%%%
%%%%%%%%%%%%%%%%%%%%%%%%%%%%%%%%%%%%%%%%%%%%%%%%%%%%%%%%%%%%%%%%%%%%%%%%%%%%%%

\section{Motivation}

Inspired by \ldbcsnb(LDBC's Social Network Benchmark)~\cite{DBLP:journals/corr/abs-2001-02299,
    ldbc_snb_docs}, a task force proposed by AntGroup~\cite{antgroup} is formed by the principal
actors in the field of financial graph-like data management with help from LDBC to design a
new benchmark, \ldbcfinbench(LDBC's Financial Benchmark). The task force intends to define a
framework which is more applicable to financial scenarios can fairly test and compare different
graph based technologies where the dataset and workload are carefully designed with their rich
practical experience of members in financial industry. \ldbcfinbench is distinguished and
characterized by the special features and patterns in financial industry compared with \ldbcsnb.

%%%%%%%%%%%%%%%%%%%%%%%%%%%%%%%%%%%%%%%%%%%%%%%%%%%%%%%%%%%%%%%%%%%%%%%%%%%%%%
%%%%%%%%%%%%%%%%%%%%%%%%%%%%%%%%%%%%%%%%%%%%%%%%%%%%%%%%%%%%%%%%%%%%%%%%%%%%%%
%%%%%%%%%%%%%%%%%%%%%%%%%%%%%%%%%%%%%%%%%%%%%%%%%%%%%%%%%%%%%%%%%%%%%%%%%%%%%%

\section{Relevance to the Industry}

\ldbcfinbench is intended to provide the following value to these relevant
stakeholders:

\begin{itemize}
    \item For \textbf{users} facing graph processing tasks in financial industry,
          \ldbcfinbench provides a recognizable scenario against which it is possible
          to compare merits of different products and technologies. By covering a wide
          variety of scales and price points, \ldbcfinbench can serve as an aid to
          technology selection.
    \item For \textbf{vendors} of graph database technology, \ldbcfinbench provides a
          checklist of features and performance characteristics that helps in product
          positioning and can serve to guide new development.
    \item For \textbf{researchers}, both industrial and academic, the \ldbcfinbench
          dataset and workload provide interesting challenges in multiple choke point
          areas, and help to compare the efficiency of existing technology in these
          areas.
\end{itemize}

Like \ldbcsnb, the technological scope of \ldbcfinbench comprises all systems that
one might conceivably use to perform financial data management tasks including
\textbf{Graph database management systems} (\eg Neo4j, TuGraph, etc.), \textbf{
    Graph processing frameworks} (\eg Giraph, Ligra, etc.), \textbf{RDF database
    systems} (\eg Virtuoso, AWS Neptune, etc.), \textbf{Relational database systems}
(\eg MySQL, Oracle, etc.), \textbf{NoSQL database systems} (\eg key-value stores
such as HBase, Redis, MongoDB, CouchDB, or even MapReduce systems like Hadoop
and Pig).

%%%%%%%%%%%%%%%%%%%%%%%%%%%%%%%%%%%%%%%%%%%%%%%%%%%%%%%%%%%%%%%%%%%%%%%%%%%%%%
%%%%%%%%%%%%%%%%%%%%%%%%%%%%%%%%%%%%%%%%%%%%%%%%%%%%%%%%%%%%%%%%%%%%%%%%%%%%%%
%%%%%%%%%%%%%%%%%%%%%%%%%%%%%%%%%%%%%%%%%%%%%%%%%%%%%%%%%%%%%%%%%%%%%%%%%%%%%%

\section{Benchmark Overview}

\subsection{Practice basis}

The task force members design \ldbcfinbench with their rich practical experience in
financial industry based on a comprehensive survey on financial scenarios including
Risk Control, AML (Anti-Money Laundering), KYC(Know Your Customer), Stock Recommendation
and so on.

\subsection{Differences between FinBench and SNB}

\ldbcsnb, one of the earlier LDBC benchmark, is modeled around the operation of a real social network site. It defines the data schema which represents a realistic social network including people and their activities during a period of time and also the workloads mimicking the different usage scenarios found in operating a real social network site. Compared with \ldbcsnb, \ldbcfinbench is characterized by the special features and patterns of the data schema and queries which represents the characteristic of financial scenarios.

\subsection{Data Schema}

The data schema for \ldbcfinbench is designed to reflect the real data in financial systems. Frequent
financial entities in real systems include accounts, medium, persons, companies, loans, etc. The
entities are vertices in the data schema while the edges reflect financial activities, \eg{fund transferred
    from one account to another}. In data schema, financial scenarios has these distinguished characteristics
compared with normal social network.
\begin{itemize}
    \item Multiple edges can exist between two vertices, \eg{ Many transfer records exist between two accounts}
    \item Dynamic attribute exists in vertex to mark entities status, \eg{an account is marked as blocked}
    \item Quantity attribute exists in edge, \eg{ Transfer edge has quantity attribute amount}
\end{itemize}

The designed data schema is specified in \autoref{sec:data-definition}.

\subsection{Workloads}

In workloads and queries, financial scenarios has these distinguished characteristics.
\begin{itemize}
    \item More tight latency, \eg{some queries need to return in less than 100ms}.
    \item Write operations updating attributes, \eg{marking an account as blocked}.
    \item Recursive Path Filtering. Some queries filter data with backward dependency
          in variable-length paths, \eg{finding all transfer paths A-[${e_1}$]->..-[${e_k}$]->B
          where the timestamp of each transfer edge ${e_i}$ in the path is larger than that of
          the previous ${e_{i-1}}$}. In this pattern, the variable length path is qualified by
          the edge quantity attributes or the aggregation in the path, either along one path
          or a set of paths.
    \item Read-write Query, which is a query sequence with a mix of reads and writes reflecting the
          complexity of financial systems. Read-write query describes a desired pattern that risk control
          policies are checked, and corresponding actions are taken before financial activities like
          transfers are actually written down to storage.
\end{itemize}

In \ldbcfinbench, there are two kinds of workloads:
\begin{itemize}
    \item Transaction Workload. It includes queries with tight latency bound, which are usually
          queries hopping a few steps from a start node. There are complex reads, simple reads, write
          operations, and read-write queries in transaction workload. Transaction workload is specified
          in \autoref{sec:transaction-workload}.
    \item Analytics Workload. It supposed to include more complicated queries,\eg{triggers and pre-computed
              values in online systems}. This part is future work which will be designed and discussed in the
          following versions. Analytics workload is specified in \autoref{sec:analytics-workload}.
\end{itemize}

%%%%%%%%%%%%%%%%%%%%%%%%%%%%%%%%%%%%%%%%%%%%%%%%%%%%%%%%%%%%%%%%%%%%%%%%%%%%%%
%%%%%%%%%%%%%%%%%%%%%%%%%%%%%%%%%%%%%%%%%%%%%%%%%%%%%%%%%%%%%%%%%%%%%%%%%%%%%%
%%%%%%%%%%%%%%%%%%%%%%%%%%%%%%%%%%%%%%%%%%%%%%%%%%%%%%%%%%%%%%%%%%%%%%%%%%%%%%

\section{Participation of Industry and Academia}

Initially, the \ldbcfinbench task force is formed by relevant actors mainly from
industry. In the process of design and development, we also received supports and
suggestions from fellows in academia. All the participants have contributed with
their experience and expertise to make this benchmark a credible effort. The names
are listed in the acknowledgments.

%%%%%%%%%%%%%%%%%%%%%%%%%%%%%%%%%%%%%%%%%%%%%%%%%%%%%%%%%%%%%%%%%%%%%%%%%%%%%%
%%%%%%%%%%%%%%%%%%%%%%%%%%%%%%%%%%%%%%%%%%%%%%%%%%%%%%%%%%%%%%%%%%%%%%%%%%%%%%
%%%%%%%%%%%%%%%%%%%%%%%%%%%%%%%%%%%%%%%%%%%%%%%%%%%%%%%%%%%%%%%%%%%%%%%%%%%%%%

\section{Software Components}

The source code of this specification and the benchmark suite are available
open-source:
\begin{itemize}
    \item \ldbcfinbench Specification: \url{https://github.com/ldbc/ldbc_finbench_docs}
    \item Transaction Workload:
          \begin{itemize}
              \item \ldbcfinbench DataGen: \url{https://github.com/ldbc/ldbc_finbench_datagen}
              \item \ldbcfinbench Driver: \url{https://github.com/ldbc/ldbc_finbench_driver}
              \item \ldbcfinbench Reference Implementation: \url{https://github.com/ldbc/ldbc_finbench_transaction_impls}
          \end{itemize}
    \item Analytics Workload: future work
\end{itemize}

Note that the \texttt{main} branch for these repositories is under development by default.
Please refer to the releases and branch started with \texttt{v} and named \texttt{vX.X.X}
for stable versions.

%%%%%%%%%%%%%%%%%%%%%%%%%%%%%%%%%%%%%%%%%%%%%%%%%%%%%%%%%%%%%%%%%%%%%%%%%%%%%%
%%%%%%%%%%%%%%%%%%%%%%%%%%%%%%%%%%%%%%%%%%%%%%%%%%%%%%%%%%%%%%%%%%%%%%%%%%%%%%
%%%%%%%%%%%%%%%%%%%%%%%%%%%%%%%%%%%%%%%%%%%%%%%%%%%%%%%%%%%%%%%%%%%%%%%%%%%%%%

\section{Related Projects}

Along with \ldbcfinbench, LDBC~\cite{DBLP:journals/sigmod/AnglesBLF0ENMKT14} provides other
benchmarks as well:

\begin{itemize}
    \item The Social Network Benchmark (SNB)~\cite{DBLP:journals/corr/abs-2001-02299} measures
          the performance of \emph{all systems relevant to linked data} operating a social network.
    \item The Semantic Publishing Benchmark (SPB)~\cite{DBLP:conf/semweb/SpasicJP16} measures
          the performance of \emph{semantic databases} operating on RDF datasets.
    \item The Graphalytics benchmark~\cite{DBLP:journals/pvldb/IosupHNHPMCCSAT16} measures the
          performance of \emph{graph analysis} operations (\eg PageRank, local clustering coefficient).
\end{itemize}