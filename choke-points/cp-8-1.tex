\snbCPSection{8.1}{LANG}{Complex patterns}

\paragraph{Description.}

A natural requirement for graph query systems is to be able to express complex
graph patterns.

\paragraph{Transitive edges.} Transitive closure-style computations are common
in graph query systems, both with fixed bounds (\eg get nodes that can be
reached through at least 3 and at most 5 \textsf{knows} edges), and without
fixed bounds (\eg get all \textsf{Messages} that a \textsf{Comment} replies to).

\paragraph{Negative edge conditions.} Some queries define \emph{negative pattern
conditions}. For example, the condition that a certain \textsf{Message} does not
have a certain \textsf{Tag} is represented in the graph as the absence of a
\textsf{hasTag} edge between the two nodes. Thus, queries looking for cases
where this condition is satisfied check for negative patterns, also known as
negative application conditions (NACs) in graph transformation
literature~\cite{DBLP:journals/fuin/HabelHT96}.

%%%%%%%%%%%%%%%%%%%%%%%%%%%%%%%%%%%%%%%%%%%%%%%%%%%%%%%%%%%%%%%%%%%%%%%%%%%%%

\IfFileExists{choke-point-query-mapping/cp-8-1}{\snbCPSection{8.1}{LANG}{Complex patterns}

\paragraph{Description.}

A natural requirement for graph query systems is to be able to express complex
graph patterns.

\paragraph{Transitive edges.} Transitive closure-style computations are common in graph query systems, both with fixed bounds
(\eg get nodes that can be reached through at least 3 and at most 5 \textsf{knows} edges),
and without fixed bounds
(\eg get all \textsf{Messages} that a \textsf{Comment} replies to).

\paragraph{Negative edge conditions.} Some queries define \emph{negative pattern conditions}. For example, the condition that a certain \textsf{Message} does not have a certain \textsf{Tag} is represented in the graph as the absence of a \textsf{hasTag} edge between the two nodes. Thus, queries looking for cases where this condition is satisfied check for negative patterns, also known as negative application conditions (NACs) in graph transformation literature~\cite{DBLP:journals/fuin/HabelHT96}.

%%%%%%%%%%%%%%%%%%%%%%%%%%%%%%%%%%%%%%%%%%%%%%%%%%%%%%%%%%%%%%%%%%%%%%%%%%%%%

\paragraph{Queries}
{\raggedright
    % \queryRefCard{interactive-complex-read-14}{IC}{14}

}}{}
