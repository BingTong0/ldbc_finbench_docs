\tpcCPSection[5.3]{5.2}{QEXE}{Overlap between outer and sub-query}

This choke point tests the ability of the execution engine to reuse results when
there is an overlap between the outer query and the sub-query. In some queries,
the correlated sub-query and the outer query have the same joins and selections.
In this case, a non-tree, rather DAG-shaped~\cite{DBLP:conf/btw/NeumannM09}
query plan would allow to execute the common parts just once, providing the
intermediate result stream to both the outer query and correlated sub-query,
which higher up in the query plan are joined together (using normal query
decorrelation rewrites). As such, the benchmark rewards systems where the
optimizer can detect this and the execution engine supports an operator that can
buffer intermediate results and provide them to multiple parent operators.

%%%%%%%%%%%%%%%%%%%%%%%%%%%%%%%%%%%%%%%%%%%%%%%%%%%%%%%%%%%%%%%%%%%%%%%%%%%%%%

\IfFileExists{choke-point-query-mapping/cp-5-2}{\tpcCPSection[5.3]{5.2}{QEXE}{Overlap between outer and sub-query}

This choke point tests the ability of the execution engine to reuse results when
there is an overlap between the outer query and the sub-query. In some queries,
the correlated sub-query and the outer query have the same joins and selections.
In this case, a non-tree, rather DAG-shaped~\cite{DBLP:conf/btw/NeumannM09}
query plan would allow to execute the common parts just once, providing the
intermediate result stream to both the outer query and correlated sub-query,
which higher up in the query plan are joined together (using normal query
decorrelation rewrites). As such, the benchmark rewards systems where the
optimizer can detect this and the execution engine supports an operator that can
buffer intermediate results and provide them to multiple parent operators.

%%%%%%%%%%%%%%%%%%%%%%%%%%%%%%%%%%%%%%%%%%%%%%%%%%%%%%%%%%%%%%%%%%%%%%%%%%%%%%

\IfFileExists{choke-point-query-mapping/cp-5-2}{\tpcCPSection[5.3]{5.2}{QEXE}{Overlap between outer and sub-query}

This choke point tests the ability of the execution engine to reuse results when
there is an overlap between the outer query and the sub-query. In some queries,
the correlated sub-query and the outer query have the same joins and selections.
In this case, a non-tree, rather DAG-shaped~\cite{DBLP:conf/btw/NeumannM09}
query plan would allow to execute the common parts just once, providing the
intermediate result stream to both the outer query and correlated sub-query,
which higher up in the query plan are joined together (using normal query
decorrelation rewrites). As such, the benchmark rewards systems where the
optimizer can detect this and the execution engine supports an operator that can
buffer intermediate results and provide them to multiple parent operators.

%%%%%%%%%%%%%%%%%%%%%%%%%%%%%%%%%%%%%%%%%%%%%%%%%%%%%%%%%%%%%%%%%%%%%%%%%%%%%%

\IfFileExists{choke-point-query-mapping/cp-5-2}{\tpcCPSection[5.3]{5.2}{QEXE}{Overlap between outer and sub-query}

This choke point tests the ability of the execution engine to reuse results when
there is an overlap between the outer query and the sub-query. In some queries,
the correlated sub-query and the outer query have the same joins and selections.
In this case, a non-tree, rather DAG-shaped~\cite{DBLP:conf/btw/NeumannM09}
query plan would allow to execute the common parts just once, providing the
intermediate result stream to both the outer query and correlated sub-query,
which higher up in the query plan are joined together (using normal query
decorrelation rewrites). As such, the benchmark rewards systems where the
optimizer can detect this and the execution engine supports an operator that can
buffer intermediate results and provide them to multiple parent operators.

%%%%%%%%%%%%%%%%%%%%%%%%%%%%%%%%%%%%%%%%%%%%%%%%%%%%%%%%%%%%%%%%%%%%%%%%%%%%%%

\IfFileExists{choke-point-query-mapping/cp-5-2}{\input{choke-point-query-mapping/cp-5-2}}{}
}{}
}{}
}{}
